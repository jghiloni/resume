\documentclass{res} 
\usepackage{hyperref}
\setlength{\textheight}{9.5in} % increase text height to fit on 1-page 

\begin{document}

\name{JOSH GHILONI\\[12pt]}     % the \\[12pt] adds a blank line after name      
\address{617 Honey Creek Road West\\
Bellville, Ohio 44813}

\address{(303) 590-5427\\
ghiloni@gmail.com}

\begin{resume}

\section{OBJECTIVE}          
    To change the world, of course. I want to help developers and operators find their value line, and provide them solutions and tools to stay above it. 
  
\section{EXPERIENCE}
   \vspace{-0.1in}	
   \begin{tabbing}
   \hspace{2.9in}\= \hspace{1.9in}\= \kill % set up two tab positions
   {\bf Principal Software Engineer} \> realtor.com \> March 2022-Present
   \end{tabbing}\vspace{-10pt}
    I was hired at RDC to be the technical lead for the SRE team. We were a small
    team of primarily young engineers, and my goal was to impart my knowledge of SRE
    practices obtained from direct experience and from teaching other teams how to 
    effectively implement the teachings in their own organizations.

    After a reorganization in September 2022, our team became focused on quality for
    the consumer-facing parts of the site. This entailed maintaining and improving 
    several internal testing tools, including k6 for performance testing, and 
    Sitespeed.io for core web vital testing. In addition, we provide reporting tools
    for product quality, including Avalability reporting, defect burndowns, and SEO 
    scores. We do this primarily by maintaining tools that extract data from various
    sources of truth and capturing summarized reports in a data warehouse.

    We have worked to migrate existing tools from a primarily AWS ECS-based deployment 
    strategy to EKS-based deployments, and I have enjoyed mentoring my team on basic
    k8s topics like RBAC and scheduling, and more advanced topics like custom resource
    definitions and operators.

    In addition to these primary responsibilities, as the technical lead and senior member
    of the team, I act as a liaison between management and the team. This involves parsing
    often ambiguous requirements from management and converting them into actionable steps
    for these young, talented engineers. I also have enjoyed the opportunity to unofficially
    mentor each of them, both individually and as a group. This involves distilling 
    complex technical requirements with them, teaching them new skills, and acting as a 
    guide as they navigate life in a corporate environment, often for the first time in 
    their careers.
   
   \begin{tabbing}
   \hspace{1.7in}\= \hspace{3.5in}\= \kill % set up two tab positions
   {\bf Founder} \> Ghiloni Ideaworx Multimedia Solutions \> 2022-Present
   \end{tabbing}\vspace{-10pt}
    Ghiloni Ideaworx is an umbrella organization for my various side endeavors, including
    intermittent independent consulting, focused usually around CI/CD solutions.

    My current project is the \href{https://github.com/gideaworx/terraform-exporter}{Terraform Exporter},
    a Go-based tool that will allow developers to write plugins to export data that can be managed 
    by Terraform into HCL files; allowing them to use their favorite UIs to create things
    like New Relic monitors and Grafana dashboards and still manage them with terraform. It is
    written in Go but allows plugins to be written in any language that can communicate over gRPC.
    The plugins can be listed in a registry for easy installation, and is viewable through a custom
    React-based webapp.
         
   \begin{tabbing}
   \hspace{2.7in}\= \hspace{1.8in}\= \kill % set up two tab positions
   {\bf Staff Solutions Architect} \> VMware, Inc. \> July 2017–February 2022
   \end{tabbing}\vspace{-10pt}
    Along with my world-class team of client-facing Solutions Architects as
    part of VMware Tanzu Labs (formerly Pivotal Labs), I helped my clients transform the way they build
    and run software platforms. My job entailed shepherding a team of client 
    platform engineers, product managers, and designers through their journey 
    in deploying and effectively using the VMware Tanzu suite and delivering value
    to their customers, which are usually in-house development teams. I focused on building 
    Site Reliability Engineering compentencies, as well fostering automation-first, 
    security-first architectures.

    From May 2020 until my departure in early 2022, I led the delivery of a small in-house SRE team as we piloted
    a new managed service offering for our customers. In this capacity, I was 
    able to take the SRE knowledge that we impart to our customer SRE teams,
    implement it myself, and work to continously improve upon that knowledge
    base. I led incident post-mortems with our customers, as well as kept them
    up to date on the health and status of their platform.

   \begin{tabbing}
   \hspace{1.9in}\= \hspace{2.5in}\= \kill % set up two tab positions
    {\bf Principal Consultant} \> ECSTeam, Inc., now part of CGI \> November 2012–July 2017
   \end{tabbing}\vspace{-10pt}
    As a boutique IT consultancy in Denver, CO, my time with ECSTeam first 
    started as an infrastructure architect for Connect for Health Colorado, 
    the ACA Exchange for the state of Colorado, based on Oracle's middleware suite. 
    
    Later, I became part of a small practice within the company focused on 
    emerging cloud technologies. As we became a Pivotal Ready Partner, I 
    participated in Pivotal Cloud Foundry "dojos" beginning in 2016. In that 
    context, I was involved with many of the same type of engagements that I 
    lead today.

   \begin{tabbing}%
   \hspace{2.9in}\= \hspace{1.5in}\= \kill % set up two tab positions
   {\bf Advisory Software Engineer}  \> IBM \> June 2003–November 2012
   \end{tabbing}\vspace{-10pt}
    My career at IBM was focused on its content management and collaboration 
    portfolio. I started in QA for the IBM Portal Personalization rules engine. 
    In 2006, I moved into a development role for the same product.

    Later, I led development efforts to integrate IBM's collaboration tools 
    Lotus Quickr and IBM Connections with its Enterprise Content Management 
    portfolio. From there, I moved to IBM SmartCloud, where I focused on 
    multi-tenancy. All development efforts were full-stack, with custom 
    Javascript client-side, J2EE back ends, with occasional dabbling in 
    other technologies (ActiveX, Flash, etc)

\section{SKILLS}          
    {\bf Programming Languages}: Go, Typescript (full-stack), Java, BASH, Python, Ruby \\
    {\bf CI/CD}: Concourse CI, CircleCI, ArgoCD, Jenkins, Github Actions \\
    {\bf Monitoring \& Logging}: InfluxDB, Prometheus, Grafana, AWS Cloudwatch

\section{CERTIFICATIONS}
    {\bf Certified Kubernetes Administrator} \\
    Certification Number CKA-2000-007976-0100 \\
    Earned May 2020

    {\bf Certified Kubernetes Application Developer} \\
    Certification Number CKAD-2000-003980-0100 \\
    Earned May 2020

    {\bf Certified Kubernetes Security Specialist} \\
    Certification Number LF-2qktgivcfa \\
    Earned July 2021
 
\section{SELECTED SPEAKING}
    {\bf April 2022} \\
    Internal realtor.com technical conference - Techcelerate \\
    On the Benefits of Building and Maintaining a Balanced Product Team

    {\bf April 2019} \\
    Cloud Foundry Summit North America \\
    \href{https://youtu.be/ydswsuidHn4}{Cracks in the Foundation}

    {\bf June 2017} \\
    Cloud Foundry Summit North America \\
    \href{https://youtu.be/dmUSqX0ELGc}{Using BOSH Addons to Customize your CF Experience}

\section{REFERENCES}
    Available upon request

\end{resume}
\end{document}
